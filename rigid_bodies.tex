\documentclass[aip,jcp,reprint,amsmath,amssymb]{revtex4-1}

\usepackage[pdftex]{graphicx}
\usepackage[]{threeparttable}

\newcommand{\mt}[1]{\boldsymbol{\mathbf{#1}}}           % matrix symbol
\newcommand{\vt}[1]{\boldsymbol{\mathbf{#1}}}           % vector symbol
\newcommand{\tr}[1]{#1^\text{t}}                        % transposition
\newcommand{\diff}[2]{\dfrac{\partial #1}{\partial #2}} % partial derivative

\newcommand{\sn}{\text{sn}}                             % Jacobi function sn
\newcommand{\cn}{\text{cn}}                             % Jacobi function cn
\newcommand{\dn}{\text{dn}}                             % Jacobi function dn

\newenvironment{smallarray}[1]                          % small arrays
{\null\,\vcenter\bgroup\scriptsize
	\renewcommand{\arraystretch}{1.5}
	\arraycolsep=.13885em
	\hbox\bgroup$\left[\array{@{}#1@{}}}
{\endarray\right]$\egroup\egroup\,\null}

\begin{document}

\title{Revisiting the Nos\'{e}-Hoover chain thermostat applied to the Rigid Bodies Molecular Dynamics}

\author{Ana J. Silveira}
\email{asilveira@plapiqui.edu.ar}
\affiliation{Planta Piloto de Ingenier\'ia Qu\'imica, PLAPIQUI, Universidad Nacional del Sur, Camino La Carrindanga Km 7-CC: 717, Bah\'ia Blanca, Argentina}

\author{Charlles R. A. Abreu}
\email{abreu@eq.ufrj.br}
\affiliation{Chemical Engineering Department, Escola de Qu\'imica, Universidade Federal do Rio de Janeiro, Rio de Janeiro, RJ 21941-909, Brazil}

\date{\today}

\begin{abstract}
In Molecular Dynamics, sets of atoms collectively behaving as rigid bodies are often used to model entire molecules or parts thereof. Such a coarse-graining strategy eliminates degrees of freedom and admits larger time steps without abandoning the atomistic character of a model. In this paper, we rely on a particular factorization of the rotation matrix to simplify the mechanics of systems containing rigid molecules. Then, we present equations of motion and devise time-reversible, measure-preserving integrators for both microcanonical (NVE) and canonical (NVT) dynamics. We also develop new expressions for computing the virial pressure of rigid-body systems under periodic boundary conditions. Finally, simulations of liquid water are employed to analyze the numerical aspects of the proposed methodology.
\end{abstract}

\maketitle

\section{Introduction}

Molecular Simulation is the computacional realization of Statistical Mechanics\cite{Tuckerman2010} that makes possible a microscopic insight into complex physical phenomena. To mimic ``real'' systems, numerical approximation are required under which certain Statistical Mechanics laws cannot be assumed to be valid also in Molecular Simulation. For instance, it has already been shown that the unavoidable truncation errors associated with the numerical integration of the equations of motion in Molecular Dynamics disrupt the equipartition of energy, and as a consequence the temperature becomes an undefined property\cite{Eastwood_2010}. In a previous paper,\citep{Abreu_2017} we observed the same problem, but regarding the partition of the kinetic energy between translational and rotational degrees of freedom in Molecular Dynamics of rigid bodies in the microcanonical ensemble. That finding brough into question the method introduced by Kamberaj and co-workers\cite{Kamberaj2005} which, assuming equipartition, couples separate chains of Nos\'{e}-Hoover thermostats on the rotational and translational degrees of freedom and fails to reproduce the specified temperature for certain time step sizes.

The relevance of rigid-body Molecular Dynamics stems from the fact that its application is not limited to molecules actually modelled as a rigid body, like the ubiquitous case of water, but also it constitues a simple coarse-grain strategy for systems that might be designed as collection of rigid bodies interconected, like proteins and nanoparticules\cite{Knorowski2012, Patra2013}. As a matter of fact, the approach of Kamberaj~\textit{et al.}\cite{Kamberaj2005}, which is implemented in the LAMMPS software package\cite{Plimpton1995}, has been applied in the simulation of small molecules\cite{Geiger_2013,Aimoli_2014,Aimoli_2014_2}, as well as in membranes~\cite{Bucior_2012} molecular motors\cite{Akimov_2012}, micelles\cite{Yan_2008} and nanoparticles\cite{Patra_2014}, DNA\cite{Fathizadeh_2012}

The Nos\'{e}-Hoover chain thermostat\cite{Martyna1992}, routinely used in Molecular Dynamics, employs the extended phase-space approach and introduces ficticius variables in order to ``control'' the kinetic energy fluctuations of the physical variables which allows to maintain the system at a specified temperature. The dynamics of those variables introduces non-Hamiltonian components which makes infeasible to carry out the backward error analysis. Altough it is not possible to devise symplectic integrator, one can rigurously obtain measure-preserving numerical solvers\cite{Sergi2001,Ezra2004,Ezra2006} and obtain numerical stability.

In this paper, we focus on the formulation of the equations of motion for rigid bodies in the canonical ensemble and introduce time reversible and measure-preserving numerical integrators as well. Our startint point is the symplectic integrator in the microcanonical ensemble we introduced in a previous paper\citep{Abreu_2017}, which employs the unit quaterion representation for the rotational degrees of freedom. Simultaneosuly coupling the translational and rotational degrees of freedom to a unique Nos\'{e}-Hoover chain thermostat, we are able to mantain the temperature of the system at the specified value. The analysis of the thermostats will be rather heuristic reliying on comparisons with results obtained employing the Hybrid Monte Carlo (HMC)\cite{Duane1987}, which is an exact method in the sense that the truncation erros are not present. We will also include results from the SHAKE\cite{Ryckaert1977} method, which is an iterative method commonly used in MD simulations, which employes lagrange multiplier to treat the holonomic constraints. unavoidable errors but we shoud be aware of the effect on the properties of interest.

The paper is organized as follows. In Sec.~\ref{sec:theory} we briefly review the equations of motion in the microcanonical ensemble, which are the basis for developing the corresponding dynamics in the canonical ensemble. In Sec.~\ref{sec:numerical_solvers} we devise the corresponding numerical integrators, whose performance and correctness we assess in Sec.~\ref{sec:numerical_results}. Finally, we present some concluding remarks in Sec.~\ref{sec:conclusion}.

\section{Theory}
\label{sec:theory}
Prompted by a particular factorization of the rotational matrix, in a previous paper\cite{Abreu_2017} we have reformulated the Hamiltonian equations of motion for rigid bodies, from which we obtained symplectic integrators in the microcanonical (NVE) ensemble. In the aforementioned paper, we introduced an analytical solution for torque-free rotations, which would be particularly useful if any backward error analysis\cite{Hairer_2006} is to be carried out. However, the  Nos\'{e}-Hoover chain thermostat introduces non-Hamiltonian components, so any backward error analysis it is not feasible anymore. Given that, in this paper, the thermostatted version of the equations of motion will be generated considering the approximate solution for the rotational motion, which is also a mathematical simplification of the formulation introduced by Miller~\textit{et al.}\cite{Miller2002}, and exhibits a precision notably similar to that achieved using the analytical solution. In the next subsection, we will skip over the mathematical details and restrict ourselves to some fundamental expressions on which the dynamics in the canonical (NVT) ensemble will be developed. For full details in the NVE ensemble the reader is referred to our previous paper.\cite{Abreu_2017}

\subsection{Microcanonical Ensemble}

When describing the motion of rigid bodies, first recall that the translational and rotational degrees of freedom are decoupled. Mathematically, the position of a particle $j$ constituent of a body, will vary with time as $\vt r_j(t) = \vt r(t) + \tr{[{\mt A}(\vt q(t))]}\vt d_j$, where $\vt r(t)$ is the center of mass position of the body, $\vt d_j$ is the particle position in the body-fixed frame of reference, and $\tr{\mt A(\vt q(t))}$ is obtained by transposing the rotational matrix $\mt A$, which in turn is a function of the unit quaternions $\vt q = \tr {[\begin{array}{cccc} q_1 & q_2 & q_3 & q_4 \end{array}]}$. In terms of $\vt q$, the dynamics must satisfy the known relation\cite{Goldstein2002} $\tr{\vt q}{\vt q} = \|\vt q\|^2 = q_1^2 + q_2^2 + q_3^2 + q_4^2 = 1$, as well as $\tr{\vt q}{\dot{\vt q}} = 0$.

The matrix $\mt A$ is computed from a given unit quaternion $\vt q$ by\cite{Allen1989,Miller2002}
\begin{equation*}
\label{eq:A_from_q}
\mt A(\vt q) = \begin{smallarray}{ccc}
q_1^2 + q_2^2 - q_3^2 - q_4^2 & 2(q_2 q_3 + q_1 q_4) & 2(q_2 q_4 - q_1 q_3) \\
2(q_2 q_3 - q_1 q_4) & q_1^2 - q_2^2 + q_3^2 - q_4^2 & 2(q_3 q_4 + q_1 q_2) \\
2(q_2 q_4 + q_1 q_3) & 2(q_3 q_4 - q_1 q_2) & q_1^2 - q_2^2 - q_3^2 + q_4^2  
\end{smallarray},
\end{equation*}
which admits the following factorization:
\begin{equation}
\label{eq:factorization_of_A}
{\mt A} = \tr{\mt B}{\mt C},
\end{equation} 
where
\begin{equation}
\label{eq:def_B_and_C}
\mt B(\vt q) = \begin{smallarray}{rrrr}
-q_2 & -q_3 & -q_4 \\
 q_1 & -q_4 &  q_3 \\
 q_4 &  q_1 & -q_2 \\
-q_3 &  q_2 &  q_1
\end{smallarray}
\; \text{and} \;
\mt C(\vt q) = \begin{smallarray}{rrrr}
-q_2 & -q_3 & -q_4 \\
 q_1 &  q_4 & -q_3 \\
-q_4 &  q_1 &  q_2 \\
 q_3 & -q_2 &  q_1
\end{smallarray}.
\end{equation}
Note that both $\mt B$ and $\mt C$ depend linearly on the Euler parameters, which makes Eq.~\eqref{eq:factorization_of_A} a convenient factorization of matrix $\mt A$. This factorization provides a relation between the three-dimensional angular velocity vector $\vt \omega$ and the four-dimensional rate of change of the quaternion components, which is
\begin{equation}
\label{eq:relation_qdot_omega}
\dot{\vt q} = \frac{1}{2} \mt B \vt \omega.
\end{equation}
The equation above avoids the introduction of a ficticious angular velocity component, as required in the formulation of Miller~\textit{et al.}\cite{Miller2002} An important feature of this relation is that it preserves the unit-norm constraint of a quaternion upon infinitesimal rotation. For the sake of brevity, the argument of a matrix-valued function (like $\mt A$, $\vt B$, or $\mt C$) might be omitted, when such argument is $\vt q$.

In the Hamiltonian approach, we employ the linear momentum $\vt p = m \dot{\vt r}$, conjugated to $\vt r$, as well as the momentum $\vt \pi= 4 \mt B \mt I \tr{\mt B} \dot{\vt q} = 2 \mt B \mt I \vt \omega$, conjugated to $\vt q$\cite{Goldstein2002}. In this equations, $m$ is the mass of the body and $\mt I$ is the inertia tensor, which is a digonal matrix in the principal frame. For a three-dimensional system consisting of $N$ non-collinear rigid bodies that interact via a potential $U(\vt r, \vt q)$, we define the Hamiltonian in terms of its generalized coordinates ($\vt r_i$ and $\vt q_i$) and conjugate momenta ($\vt p_i$ and $\vt \pi_i$), as follows:~\cite{Abreu_2017}

\begin{equation}
\label{eq:H_NVE}
\mathcal{H} = U(\vt r,\vt q) + \sum_{i=1}^N \left(\frac{\tr{\vt p}_i{\vt p}_i}{2m_i} + \frac{\tr{\vt \pi}_i {\mt B}_i {\mt I}_i^{-1} \tr{\mt B}_i {\vt \pi}_i}{8}\right),
\end{equation}
where the index $i$ ranging from $1$ to $N$ indicates the quantities related to each body.

The equations of motion are obtained from the Hamiltonian by $\dot{\vt r}_i = \partial \mathcal{H} / \partial \vt p_i$, $\dot{\vt p}_i = -\partial \mathcal{H} / \partial \vt r_i$, $\dot{\vt q}_i = \partial \mathcal{H} / \partial \vt \pi_i$, and $\dot{\vt \pi}_i = -\partial \mathcal{H} / \partial \vt q_i$.\cite{Goldstein2002}. For a rigid body composed of $n_p$ individual particles, the derivative $\partial U/\partial \vt r$ is obtained by
\[
\diff{U}{\vt r_i} = -\vt F_i = -\sum_{j=1}^{n_p} {\vt F_j},
\]
where $\vt F_j$ is the force applied on particle $j$, expressed in the space-fixed frame of reference and thus $\vt F_i$ is the resultant force on the body. As demonstrated in Abreu~\textit{et al.},\citep{Abreu_2017} $\partial{U}/\partial \vt q_i$ is given by
\[
\diff{U}{\vt q_i} = -2 \mt C_i \vt \tau_i = -2 \mt C_i \sum_{j=1}^{n_p} {\vt \delta_j} \times {\vt F_j},
\]
where $\vt \tau_i$ is the resultant torque exerted on the body and ${\vt \delta}_j$ is the position of the particle $j$, both expressed in the space-fixed frame.  The position ${\vt \delta}_j$ is related to ${\vt d}_j$ by ${\vt \delta}_j = \tr{\mt A} {\vt d}_j$.\cite{Goldstein2002} After carrying out the required differentiations of $\mathcal{H}$  we obtain the following Hamiltonian system of equations of motion for a rigid body:
\begin{subequations}
\label{eq:EDO_system}
\begin{align}
&\dot{\vt r}_i = \frac{1}{m_i} \vt p_i \\
&\dot{\vt p}_i = \mt F_i \\
&\dot{\vt q}_i = \frac{1}{2} \mt B_i \vt \omega_i \label{eq:EDO_q} \\
&\dot{\vt \pi}_i = \frac{1}{2} \mt \Omega_i \vt \omega_i + 2 \mt C_i \vt \tau_i \label{eq:EDO_pi}
\end{align}
\end{subequations}
with $\mt \Omega_i = \mt B_i(\vt \pi_i)$.

The approximate solution for a complete rotation is obtained by opening the products $\mt B \vt \omega$ and $\mt \Omega \vt \omega$ employing each entry of $\vt \omega$: 
\begin{equation}
\label{eq:omega_entry}
\omega_k = \frac{\tr{\vt \pi} \hat{\mt B}_k \vt q}{2 I_k},
\end{equation}
where $\hat{\mt B}_k$ is one of the following permutation matrices:
\[
\hat{\mt B}_1 = \begin{smallarray}{cccc}
 0 & \text{-}1 &  0 &  0 \\
 1 &  0 &  0 &  0 \\
 0 &  0 &  0 &  1 \\
 0 &  0 & \text{-}1 &  0 \\
\end{smallarray} \quad
\hat{\mt B}_2 = \begin{smallarray}{cccc}
 0 &  0 & \text{-}1 &  0 \\
 0 &  0 &  0 & \text{-}1 \\
 1 &  0 &  0 &  0 \\
 0 &  1 &  0 &  0 \\
\end{smallarray} \quad
\hat{\mt B}_3 = \begin{smallarray}{cccc}
 0 &  0 &  0 & \text{-}1 \\
 0 &  0 &  1 &  0 \\
 0 & \text{-}1 &  0 &  0 \\
 1 &  0 &  0 &  0 \\
\end{smallarray}
\]
The new form of the equations of motion in Eqs.~\eqref{eq:EDO_q} and \eqref{eq:EDO_pi} are then:
\begin{subequations}
\label{eq:EDO_system_alternative}
\begin{align}
&\dot{\vt q}_i = \frac{1}{2} \left( \sum_{k=1}^3 \omega_{ik} \hat{\mt B}_k \right) \vt q_i \label{eq:edo_q} \\
&\dot{\vt \pi}_i = \frac{1}{2} \left( \sum_{k=1}^3 \omega_{ik} \hat{\mt B}_k \right) \vt \pi_i + 2 \mt C_i \vt \tau_i \label{eq:edo_pi}
\end{align}
\end{subequations}

In this context, the Hamiltonian of the rigid body can be rewritten as
\begin{equation}
\label{eq:H_split_omega}
\mathcal{H} =  U(\vt r, \vt q) +  \sum_{i=1}^N \left( \frac{\tr{\vt p_i} \vt p_i}{2m_i} + \frac{1}{2} \sum_{k=1}^3 I_{ik} \omega_{ik}^2\right),
\end{equation}
from which we recognize partial Hamiltonians, each of one generates equations of motion with known analytical solutions. In particular, we have $\mathcal{H}_b = U(\vt r, \vt q)$, $\mathcal{H}_{t} = \frac{1}{2m}\tr{\vt p} \vt p$, and $\mathcal{H}_k = (1/2) I_k \omega_k^2$. Of coarse, $\mathcal{H} = \mathcal{H}_b + \mathcal{H}_t + \sum_{k=1}^3 \mathcal{H}_k$. The NVE dynamics of this system is governed by a Liouville operator given by\cite{Abreu_2017}
\begin{equation}
\label{eq:full_operator}
i\!L_{NVE} = i\!L_t + i\!L_b + i\!L_r,
\end{equation}
where the individual operators are defined as follows:
\[
\begin{split}
&boost:\\
&i\!L_b = \sum_{i=1}^N \left( \tr{\vt F}_i \diff{}{\vt p_i} + 2 \tr{\vt \tau}_i \tr{\mt C}_i \diff{}{\vt \pi_i} \right), \\
&translation: \\
& i\!L_t = \sum_{i=1}^N \frac{\tr{\vt p}_i}{m_i}\diff{}{\vt r_i}, \quad \text{and} \\
&uniaxial \; rotation:\\
& i\!L_k = \frac{1}{2} \sum_{i=1}^N {\omega_i}_k \left( \tr{\vt q}_i\tr{\hat{\mt B}_k} \diff{}{\vt q_i} + \tr{\vt \pi_i}\tr{\hat{\mt B}_k} \diff{}{\vt \pi_i} \right),
\end{split}
\]
using $i\!L_r = \sum_{k=1}^3 i\!L_k$ for a complete torque-free rotation. As explained in our previous paper,\cite{Abreu2017} those individual operators can be applied analytically according to
\[
\begin{split}
&boost:\\
&{\vt p} = {\vt p}_0 + {\vt F}_0 t \\
&{\vt \pi} = {\vt \pi}_0 +  2 {\mt C}_0 {\vt \tau}_0 t\\
&translation:\\
&{\vt r} = {\vt r}_0 + (t/m) {\vt p}_0\\
&uniaxial \; rotation:\\
&{\vt q}(t) = \cos\left(\frac{\omega_k t}{2}\right) \vt q_0 + \sin\left(\frac{\omega_k t}{2}\right) \hat{\mt B}_k \vt q_0 \\
&{\vt \pi}(t) = \cos\left(\frac{\omega_k t}{2}\right) \vt \pi_0 + \sin\left(\frac{\omega_k t}{2}\right) \hat{\mt B}_k \vt \pi_0
\end{split}
\]

Applying the Trotter-susz The splitting of the Liouville operator of Eq.~\eqref{eq:full_operator} we employ in this paper is
\begin{equation}
\label{eq:trotter_splitting_NVE}
e^{\Delta t i\!L_{NVE}} = e^{\frac{\Delta t}{2} i\!L_b} e^{\Delta t i\!L_r} e^{\Delta t i\!L_t} e^{\frac{\Delta t}{2} i\!L_b},
\end{equation}
Update of forces and torques is required only after the free translation and free rotation have been carried out. Following Miller \textit{et al}.,\cite{Miller2002} each complete rotation is splitted as
\begin{align*}
\label{eq:splitting_rot}
e^{\Delta t i\!L_r} =  e^{\frac{\Delta t}{2} i\!L_3} e^{\frac{\Delta t}{2} i\!L_2} e^{\Delta t i\!L_1} e^{\frac{\Delta t}{2} i\!L_2} e^{\frac{\Delta t}{2} i\!L_3},
\end{align*}
where ${\Delta t}$ is the time step size.

equipartion of energy

\subsection{Canonical Ensemble}
\label{sec:canonical}

Knowledge about the dynamics and statistical mechanics of non-Hamiltonian systems has advanced considerably in recent years.\cite{Tuckerman_1999, Tuckerman2001, Sergi2001, Sergi2003, Ezra2004, Sergi2004, Ezra2006, Sergi2010b} In this context, invaluable tools have been devised for the development and analysis of simulation methods based on the concept of extended phase space. Sergi and Ferrario\cite{Sergi2001} describe a simple way of systematically generating conservative trajectories in an arbitrary phase space. If vector $\vt x$ represents a phase-space point, $H(\vt x)$ is a scalar field, and $\boldsymbol{\mathcal B}(\vt x)$ is a matrix-valued function, then the equation of motion
\begin{equation} \label{eq:eq_of_motion}
\dot{\vt x} = \boldsymbol{\mathcal B}\diff{H}{\vt x}
\end{equation}
will generate a flow that conserves the value of $H$, provided that $\boldsymbol{\mathcal B}$ is skew-symmetric (that is, $\tr{ \boldsymbol{ \mathcal B }} = -\boldsymbol{ \mathcal B }$). A generalized Liouville operator can then be defined as\cite{Sergi2004}
\[
i\!L = \tr{\dot{\vt x}}\diff{}{\vt x} = \tr{\left(\diff{H}{\vt x}\right)} \tr{\boldsymbol{\mathcal B}} \diff{}{\vt x}.
\]

The system might exhibit a non-vanishing phase-space compressibility $\kappa(\vt x) = \sum_j \partial \dot{x}_j/\partial x_j$ if any variable in $\vt x$ affects its own time-derivative. According to Tuckerman and coworkers,\cite{Tuckerman_1999, Tuckerman2001} although the phase-space volume element $d\vt x$ is no longer conserved, there exits an invariant measure $e^{-w(\vt x)}d\vt x$, where $w(\vt x)$ is the indefinite time integral of $\kappa(\vt x)$. Using the generalized Liouville operator, such relation between $\kappa$ and $w$ can be written as
\begin{equation}
\label{eq:relation_kappa_w}
\kappa = \dot w = i\!L w.
\end{equation}

We employ the technique of Sergi and Ferrario\cite{Sergi2001} to couple a single Nos\'e-Hoover chain\cite{Martyna1992} (NHC) composed of $M$ thermostats to a system of $N$ rigid bodies. This is simpler than the approach of Kamberaj \textit{et al}.,\cite{Kamberaj2005} who employed two thermostat chains independently coupled to the translational and rotational degrees of freedom. To this end, we consider extra coordinates $\eta_j$ and their conjugate momenta $p_{\eta_j}$, for $j$ ranging from $1$ to $M$. The conserved ``extended Hamiltonian'' in this case is
\begin{equation}
\label{eq:H_nvt}
H = \mathcal{H}(\vt r,\vt p,\vt q,\vt \pi) + \sum_{j=1}^{M}\frac{p_{\eta_j}^2}{2Q_j} + L k_b T\eta_1 + k_b T\sum_{j=2}^M \eta_j,
\end{equation}
where $\mathcal H$ is the same as in Eq.~\eqref{eq:H_NVE}, $k_b T$ is the Boltzmann constant multiplied by the system temperature, $L$ is a constant to be determined, and each $Q_j$ is an inertial parameter of thermostat $j$. As recommended in Ref.~\onlinecite{Martyna1992}, one can make $Q_1 = L k_b T t_d^2$ and $Q_j = k_b T t_d^2$ for $j \geq 2$, where $t_d$ is a characteristic time scale of the system. The equations of motion regarding generalized coordinates are identical to those of a Hamiltonian system, i.e. $\dot{\vt r}_i = {\partial H}/{\partial \vt p_i}$ and $\dot{\vt q}_i = {\partial H}/{\partial \vt \pi_i}$ for every particle $i$ and $\dot{\eta}_j = {\partial H}/{\partial {p_\eta}_j}$ for every thermostat $j$. However, the equations of motion for generalized momenta contain non-Hamiltonian terms. For instance, the first thermostat of the chain acts on the translational and rotational momenta of each particle $i$ as
\[
\begin{split}
\dot{\vt p}_i &= -\diff{H}{\vt r_i} - {\vt p}_i \diff{H}{p_{\eta_1}} \quad \text{and} \\
\dot{\vt \pi}_i &= -\diff{H}{\vt q_i} - {\vt \pi}_i \diff{H}{p_{\eta_1}}.
\end{split}
\]

In the right-hand side of each equation above, the first term fulfills the skew-symmetry condition with respect to the corresponding coordinate while the second term enacts the coupling with the mentioned thermostat. In turn, the equation of motion for $p_{\eta_1}$ must contain terms to ensure skew-symmetry with respect to the equations already shown, along with a term regarding the action of a second thermostat. The result is
\[
{\dot p}_{\eta_1} = \sum_{i=1}^N \left( \tr{\vt p_i} \diff{H}{\vt p_i} + \tr{\vt \pi_i} \diff{H}{\vt \pi_i}\right) - \diff{H}{\eta_1} - p_{\eta_1} \diff{H}{p_{\eta_2}}.
\]

Each additional thermostat will then act on the preceding one, and be acted upon by the succeeding one. Therefore, for $j$ from $2$ to $M-1$, the equations of motion are
\[
{\dot p}_{\eta_j} = -\diff{H}{\eta_j} + p_{\eta_{j-1}} \diff{H}{p_{\eta_{j-1}}} - p_{\eta_j} \diff{H}{p_{\eta_{j+1}}}.
\]

Given that the last thermostat will just act on the previous one, its equation of motion can be obtained by simply replacing $j = M$ and removing the last term in the equation above. Therefore, the equations of motion for a system of rigid bodies coupled to a single Nos\'e-Hoover chain become
\begin{subequations}
\label{eq:nhc_system}
\begin{align}
&\dot{\vt r}_i = \frac{{\vt p}_i}{m_i} \\ 
&\dot{\vt p}_i = {\vt F}_i - \alpha_1 \vt p_i \label{eq:nhc_p} \\
&\dot{\vt q}_i = \frac{1}{2} \mt B_i \vt \omega_i \label{eq:nhc_q} \\
&\dot{\vt \pi}_i = \frac{1}{2} \mt \Omega_i \vt \omega_i + 2 \mt C_i \vt \tau_i - \alpha_1 \vt \pi_i \label{eq:nhc_pi} \\
&\dot{\eta}_j = \alpha_j \label{eq:nhc_eta} \\
&{\dot p}_{\eta_j} = G_j - \alpha_{j+1} p_{\eta_j}  \label{eq:nhc_p_eta}
\end{align}
\end{subequations}

In the equations above, ${\vt \omega}_i$ $=$ $\frac{1}{2} {\mt I}_i^{-1} \tr{\mt B}_i {\vt \pi}_i$, $\alpha_j$ $=$ ${p_{\eta_j}}/{Q_j}$ for $j \leq M$ and $\alpha_{M+1} = 0$, while
\begin{align*}
&G_1 = \sum_{i=1}^N \left( \frac{\tr{\vt p}_i{\vt p}_i}{m_i} + \tr{\vt \omega}_i \mt I_i \vt \omega_i \right) - L k_b T \quad \text{and}\\
&G_j = \frac{p_{\eta_{j-1}}^2}{Q_{j-1}} - k_b T \qquad \text{for} \; j > 1.
\end{align*}

Note that the $3N$ momentum components in Eq.~\eqref{eq:nhc_p}, the $4N$ components in Eq.~\eqref{eq:nhc_pi}, and the thermostat momenta in Eq.~\eqref{eq:nhc_p_eta} affect their own time-derivatives. Thus, the thermostatted system has a phase-space compressibility $\kappa = -7N \alpha_1 - \sum_j \alpha_{j+1}$. Resorting to Eq.~\eqref{eq:nhc_eta} and the fact that $\alpha_{M+1} = 0$, the function $w(\vt x)$ that determines the invariant phase-space measure is given by
\begin{equation}
\label{eq:nhc_measure}
w = -7N \eta_1 - \sum_{j=2}^M \eta_j.
\end{equation}

Finally, analysis of the probability distribution generated by the canonical equations of motion can be carried out via the generalized phase-space analysis of Tuckerman \textit{et al}.\cite{Tuckerman2001} For this, conservation laws must be identified and driven variables must be eliminated. For a system without external forces, the extended energy $H$ and the vector quantity $e^{\eta_1}\sum_{i=1}^N {\vt p}_i = \vt K$ are integrals of motion, while the center-of-mass position can be considered as a driven variable.\cite{Tuckerman2001} Moreover, $2N$ equations must be eliminated from the system due to the constraints involving $\vt q_i$ and $\vt \pi_i$ for all $i$. In the general case, by taking all these facts into account and applying the analysis of Ref.~\onlinecite{Tuckerman2001}, we can deduce that the metric determinant factor of the effective phase space is $\sqrt{g} = e^{(6N-2) \eta_1 + \sum_{j=2}^M \eta_j}$ and that the correct canonical distribution is attained if one makes $L = 6N$. In the particular (and rather common) situation in which one sets $\vt P = \vt 0$ at the beginning of a simulation, one must make $L = 6N - 3$ because all three elements of $\vt P$ will be conserved.\cite{Martyna1994}

\subsection{Reversible, Measure-Preserving Integration of the Canonical Equations of Motion}
\label{sec:numerical_solvers}
As in the Hamiltonian case, the evolution of a phase-space function $f(\vt r, \vt p, \vt q, \vt \pi, \vt \eta,{\vt p}_\eta)$ is given by the relation $f(t) = e^{t i\!L}f_0$, but now in the context of the generalized Liouville theorem.\cite{Tuckerman_1999, Tuckerman2001} Thus, we can employ the Trotter-Suzuki splitting to devise a time-reversible algorithm for integrating the equations of motion. In the microcanonical case, each separate term of the Liouville operator is Hamiltonian and, therefore, each factor of the splitting is a symplectic propagator. As a result, the whole propagator is symplectic and thus preserves the phase-space volume element. It is widely recognized that this property is important for numerical stability.\cite{Skeel1997} In the case of non-Hamiltonian systems, Ezra\cite{Ezra2006} introduced a systematic method for deriving accurate integrators which are not only time-reversible, but also preserve the phase-space measure associated with the original equations of motion. In short, a factorization scheme will preserve a given measure $e^{-w(\vt x)}d\vt x$ if each individual propagator preserves the same measure. As the author has shown,\cite{Ezra2006} for a particular splitting $i\!L = \sum_k i\!L^{(k)}$, measure preservation will take place if $\sum_j \tfrac{\partial}{\partial x_j} [e^{-w(\vt x)}\dot{x}_j^{(k)}] = 0$ for all $k$. Here, $\dot{\vt x}^{(k)}$ refers to the system of equations obtained from the individual operator $i\!L^{(k)}$, which might display its own compressibility $\kappa^{(k)} = \sum_j \partial \dot{x}_j^{(k)}/\partial x_j$. By means of the product rule, it is straightforward to devise an equivalent but more meaningful criterion, which is
\begin{equation}
\label{eq:split_kappa_w}
\kappa^{(k)} = i\!L^{(k)} w.
\end{equation}

Therefore, if the function $w(\vt x)$ that derives from Eq.~\eqref{eq:relation_kappa_w} also satisfies Eq.~\eqref{eq:split_kappa_w} for all $k$, then the corresponding factorization will preserve the measure $e^{-w(\vt x)}d\vt x$. Here, we propose and compare two distinct measure-preserving integrators for a rigid-body system coupled to a single thermostat chain. In the first one, the generalized Liouville operator is written as $i\!L = i\!L_\text{NVE} + i\!L_\text{NHC}$, where $i\!L_\text{NVE}$ is defined as in Eq.~\eqref{eq:full_operator} and, as it follows from Eq.~\eqref{eq:nhc_system},
\begin{equation}
\label{eq:iL_NHC}
\begin{split}
i\!L_\text{NHC} = &-\alpha_1 \sum_{i=1}^N \left( \tr{\vt p}_i \diff{}{\vt p_i} + \tr{\vt \pi}_i \diff{}{\vt \pi_i}\right) + \\
&+ \sum_{j=1}^{M} \left[\alpha_j \diff{}{\eta_j} + (G_j - \alpha_{j+1} p_{\eta_j}) \diff{}{p_{\eta_j}}\right].
\end{split}
\end{equation}

Then, by retaining the NVE factorization presented in Eq.~\eqref{eq:trotter_splitting_NVE}, we write the full exponential operator for each time step as
\begin{equation}
\label{eq:trotter_splitting_NHC}
e^{\Delta t i\!L} = e^{\frac{\Delta t}{2} i\!L_\text{NHC}} e^{\frac{\Delta t}{2} i\!L_b} e^{\Delta t i\!L_r} e^{\Delta t i\!L_t}  e^{\frac{\Delta t}{2} i\!L_b} e^{\frac{\Delta t}{2} i\!L_\text{NHC}}.
\end{equation}

The boost, translation, and rotation propagators already satisfy the criterion in Eq.~\eqref{eq:split_kappa_w} with the function $w$ expressed in Eq.~\eqref{eq:nhc_measure} (they are all incompressible and keep $w$ unchanged). The NHC propagator is further split into factors that also satisfy Eq.~\eqref{eq:split_kappa_w}, whichnare given by

\begin{subequations}
\begin{align}
i\!L_\text{NHC}^{(0)} &= -\alpha_1 \sum_{i=1}^N \left( \tr{\vt p}_i \diff{}{\vt p_i} + \tr{\vt \pi}_i \diff{}{\vt \pi_i}\right) + \alpha_1 \diff{}{\eta_1}, \\
i\!L_\text{NHC}^{(j)} &= (G_j - \alpha_{j+1} p_{\eta_j}) \diff{}{p_{\eta_j}} + \alpha_{j+1} \diff{}{\eta_{j+1}}, \label{eq:iL_NHC_j} \\
i\!L_\text{NHC}^{(M)} &= G_M \diff{}{p_{\eta_M}},
\end{align}
\end{subequations}
with $j$ varying from $1$ to $M-1$ in Eq.~\eqref{eq:iL_NHC_j}. Then, each propagator $e^{\frac{\Delta t}{2} i\!L_\text{NHC}}$ present in Eq.~\eqref{eq:iL_NHC} is factorized as
\[
e^{\frac{\Delta t}{2} i\!L_\text{NHC}} = \Bigg[ \prod_{j=M}^1 e^{\frac{\Delta t}{4} i\!L_\text{NHC}^{(j)} } \Bigg] e^{\frac{\Delta t}{2} i\!L_\text{NHC}^{(0)} } \Bigg[ \prod_{j=1}^M e^{\frac{\Delta t}{4} i\!L_\text{NHC}^{(j)} } \Bigg].
\]

The action of the propagator $e^{t i\!L_\text{NHC}^{(0)}}$ is the solution of the equations $\dot{\vt p}_i = -\alpha_1 \vt p_i$ and $\dot{\vt \pi}_i = -\alpha_1 \vt \pi_i$ for all $i \in [1,N]$ and $\dot{\eta}_1 = \alpha_1$, with all other variables remaining constant at their initial values. This system can be solved analytically, which results in
\begin{align*}
&\vt p_i(t) = \vt p_i^0 e^{-\alpha_1 t} \\
&\vt \pi_i(t) = \vt \pi_i^0 e^{-\alpha_1 t} \\
&\eta_1(t) = \eta_1^0 + \alpha_1 t
\end{align*}

In turn, the propagator $e^{t i\!L_\text{NHC}^{(j)}}$ defined in Eq.~\eqref{eq:iL_NHC_j} behaves according to the equations $\dot{p}_{\eta_j} = G_j - \alpha_{j+1} p_{\eta_j}$ and $\dot{\eta}_{j+1} = \alpha_{j+1}$, with all variables other than $p_{\eta_j}$ and $\eta_{j+1}$ kept unchanged, whose analytical solution is
\begin{align*}
&p_{\eta_j}(t) = p_{\eta_j}^0 + \Big( G_j - \alpha_{j+1} p_{\eta_j}^0 \Big) \phi\left(\alpha_{j+1} t\right) t \\
&\eta_{j+1}(t) = \eta_{j+1}^0 + \alpha_{j+1} t
\end{align*}
where $\phi(x) = (1-e^{-x})/x$. We might note that, in order to avoid numerical issues with $x \rightarrow 0$, we evaluate $\phi(x)$ as $\sum_{n=0}^3 {(-1)^n x^n}/{(n+1)!}$ whenever $|x| \leq 10^{-4}$. The solution above is equivalent to that found in Ref.~\onlinecite{Martyna1996}.

Finally, the propagator $e^{t i\!L_\text{NHC}^{(M)}}$ involves a unique differential equation $\dot{p}_{\eta_M} = G_M$, whose solution is
\[
p_{\eta_M}(t) = p_{\eta_M}^0 + G_M t.
\]

The second splitting scheme we propose here relies on the definition of the operator
\begin{align*}
i\!L_b^\ast &= \sum_{i=1}^N \left( \tr{\vt F}_i - \alpha_1 \tr{\vt p}_i \right) \diff{}{\vt p_i} \\
&+ \sum_{i=1}^N \left(2 \tr{\vt \tau}_i \tr{\mt C}_i - \alpha_1 \tr{\vt \pi}_i \right) \diff{}{\vt \pi_i} + \alpha_1 \diff{}{\eta_1},
\end{align*}
with which analytical solutions can also be derived for the system of equations $\dot{\vt p}_i = {\vt F}_i - \alpha_1 \vt p_i$ and $\dot{\vt \pi}_i = 2 \mt C_i \vt \tau_i - \alpha_1 \vt \pi_i$ for all $i \in [1,N]$ if all other variables are constant. These equations are equivalent in form to the ones solved above, so we get
\begin{align*}
&{\vt p}_i(t) = {\vt p}_i^0 + \left({\vt F}_i - \alpha_1 {\vt p}_i^0 \right) \phi\left(\alpha_1 t \right) t \\
&{\vt \pi}_i(t) = {\vt \pi}_i^0 + \left(2 {\mt C}_i{\vt \tau}_i - \alpha_1 {\vt \pi}_i^0 \right) \phi\left(\alpha_1 t \right) t \\
&\eta_1(t) = \eta_1^0 + \alpha_1 t
\end{align*}

In the new propagator, $e^{\frac{\Delta t}{2} i\!L_\text{NHC}^{(0)} }$ is removed:
\[
e^{\frac{\Delta t}{2} i\!L_\text{NHC}^\ast} = \Bigg[ \prod_{j=M}^2 e^{\frac{\Delta t}{4} i\!L_\text{NHC}^{(j)} } \Bigg] e^{\frac{\Delta t}{2} i\!L_\text{NHC}^{(1)} } \Bigg[ \prod_{j=2}^M e^{\frac{\Delta t}{4} i\!L_\text{NHC}^{(j)} } \Bigg].
\]

We remark that the integration scheme proposed by Kamberaj \text{et al}.\cite{Kamberaj2005} does not fulfill this criterion. 
Finally, the complete scheme is
\begin{equation}
\label{eq:modified_splitting}
\begin{split}
e^{\Delta t i\!L} = e^{\frac{\Delta t}{2} i\!L_\text{NHC}^\ast} e^{\frac{\Delta t}{2} i\!L_b^\ast} e^{\Delta t i\!L_r} e^{\Delta t i\!L_t}  e^{\frac{\Delta t}{2} i\!L_b^\ast} e^{\frac{\Delta t}{2} i\!L_\text{NHC}^\ast}.
\end{split}
\end{equation}

We end this section by noting that the term that premultiplies the extended phase-space volume in the invariant-measure above is a function of $\vt \eta$. Given that the time evolution of the physical variables and the thermostat momenta does not depend on $\vt \eta$, we might conclude that, in this case, the measure-preserving condition for every individual operator can actually be relaxed. However...the energy drift...

\subsection{Sampling Validation}
\label{sec:samplingvalidation}

In this subsection we briefly review a simple method that serves for checking the consistency between the sampled and theoretical energy distributions. The procedures implemented by Shirts ~\cite{Shirts2013} rely on quantitative statistical analysis of fitting, and it is possible to perform linear and nonlinear fitting or a maximum likelihood approach too. Fist recall that in the NVT ensemble the probability of observing an energy E is given by
\[
P(E/\beta) = Q(\beta)^{-1}\Omega(E)\exp\left(-\beta E\right),
\]
where $ \Omega(E) $ is the density of states, which does not depend on $ \beta $, and $ Q(\beta) $  is the canonical partition function, linked to the Helmholtz Free Energy $ (A) $ by $A = -\beta^{-1}\ln(Q)$. 

Taking the ratio of the probability distributions at different temperatures, the unknown density of states cancels out and we get
\begin{equation}
\frac{P(E/\beta_2)}{P(E/\beta_1)} = \exp\left[(\beta_2 A_2 - \beta_1 A_1) - (\beta_2 - \beta_1)E\right].
\end{equation}
Taking the logarithm of the equation above, we obtain
\begin{equation}
\label{eq:ln_probability_ratio}
\ln\frac{P(E/\beta_2)}{P(E/\beta_1)} = \left(\beta_2 A_2 - \beta_1 A_1\right) - \left(\beta_2 - \beta_1\right)E,
\end{equation}
which is the basis for the statistical analysis carried out in the ensemble validation that determines how many standard deviations the calculated slope is from the specified one,  $ \beta_2 - \beta_1 $. Note that it is independent of the unknown free energies. For a guidance on how to choose the temperature gap the reader is referred to the aforementioned paper.

Furthermore,  it is possible to examine the consistency between the sampled kinetic energy $K$ and the corresponding Maxwell-Boltzmann distribution, which is given by the following expression:
\begin{equation}
\label{eq:mb}
P(K) = \frac{\beta}{\sqrt{N_{dof}\pi}} \exp\left[-\frac{\left(\beta K - \frac{N_{dof}}{2}\right)^2}{N_{dof}} \right],
\end{equation}
where $N_{dof}$ is the number of degrees of freedom. Eq.~\eqref{eq:mb} is a normal distribution, with average  $ \langle K \rangle = N_{dof}/2\beta $ and variance $\sigma^2 = N_{dof}/2\beta^2$.

\section{Results}
\label{sec:numerical_results}

CÓDIGO: CHECAR FACTORIZACIÓN Y DELTA T DE LOS OPERADORES DE NHC

In this section, we analyze the numerical integrators introduced in Subsection~\ref{sec:numerical_solvers}, and compare the results with those obtained employing the method of Kamberaj~\textit{et al.}\cite{Kamberaj2005}, which is implemented in the LAMMPS\cite{Plimpton1995} software package. For that, we considered a system of 900 TIP3P\cite{Jorgensen_1983} water molecules in a cubic box of length {\AA} subject to periodic boundary conditions. The intermolecular interactions were truncated at 9 \AA and to avoid a discontinuity in the force, we applied the so-called shifted-force (SF) method\cite{Allen1989, Fennell2006, Toxvaerd_2011} to treat both the Coulombic and the 12-6 Lennard-Jones (LJ) interactions.

It is not the purpose of this work to propose an optimal factorization scheme for the NVT dynamics, but rather to prove that a unique thermostat chain is capable of maintaining the average temperature at its specified value while providing the proper sampling of the canonical ensemble. First we present some results as first mented. That is, the proposed methods coupling to a single thermostat, while in Kamberaj the translational and rotational degrees of freedom are separatedly coupled. In Table~\ref{table:new_results}, it can be seen that the integrators proposed here, which are represented by the Eqs.~\eqref{eq:trotter_splitting_NHC} and ~\eqref{eq:modified_splitting}, systemcatically achive the specifed temperature, regardless of the time step size. By contrast, the numerical solver introduced by Kamberaj and co-workers\cite{Kamberaj2005} fails to yield the specified temperature for even the smaller time step size of 1 fs.
 
Resultados sobre o checkensemble: We report the results corresponding to that method, and since it is instructive to have a graphical insight, we also show some figures of the linear fitting. Given that the potential and kinetic energies are separable, the analysis can be made separately as follows.This last approach does not involve any binning of the energies, hence no discretization error is present. m-b distribution valid when equipartition of energy is satisfied.

Recalling that in the NVT dynamics the conserved quantity is the extended energy $H$ of Eq.~\eqref{eq:H_nvt}, in Table~\ref{table:new_results_2} we report the energy drift, denoted by $R$, as the the rate of change of $H$ (kcal/mol) with time (ns), which is the angular coefficient obtained from linear regression fitting  between those variables. Except for the case of $\Delta$ = 1 fs, there is also a clear correspondece between an incresing $R$ with the failure in keeping the temperature at its specified value.

\begin{table}
	\caption{New results involving Kamberaj \text{et al}.\cite{Kamberaj2005}}
	\label{table:new_results}
	\begin{ruledtabular}
		\begin{tabular}{cccc}
			& \multicolumn{3}{c}{ Temperature (K) } \\
			\cline{2-4}
			$\Delta t$/fs & Proposed & Alternative & Kamberaj \text{et al}. \\
			\hline
			1.0 & $300.01 \pm 0.02$ & $300.00 \pm 0.02$ & $299.52 \pm 0.02$ \\
			2.0 & $300.01 \pm 0.03$ & $299.99 \pm 0.03$ & $298.24 \pm 0.03$ \\
			3.0 & $299.97 \pm 0.04$ & $299.98 \pm 0.04$ & $295.94 \pm 0.04$
		\end{tabular}
	\end{ruledtabular}
\end{table}

\begin{table}
	\caption{New results involving Kamberaj \text{et al}.\cite{Kamberaj2005}}
	\label{table:new_results_2}
	\begin{ruledtabular}
		\begin{tabular}{cccc}
			& \multicolumn{3}{c}{Drift rate [(kcal/mol)/ns]} \\
			\cline{2-4}
			$\Delta t$/fs & Proposed & Alternative & Kamberaj \text{et al}. \\
			\hline
			1.0 & $0.0328$ & $0.108$ & $0.023$ \\
			2.0 & $0.597$  & $1.31$  & $17.0$ \\
			3.0 & $3.44$   & $2.53$  & $53.2$
		\end{tabular}
	\end{ruledtabular}
\end{table}


\begin{table*}
\caption{Results.....}
\label{table:proposed_method}
\begin{ruledtabular}
\begin{tabular}{cccccccc}
$\Delta t$ (fs) & $T$ (K) & $\langle U\rangle$ (kcal/mol) & $\langle K\rangle$ (kcal/mol) & $\langle K_t\rangle$ (kcal/mol) & $\langle K_r\rangle$ (kcal/mol) & P (atm) & $R$ (kcal/mol.ns) \\

\hline
\multicolumn{8}{c}{Proposed Method} \\
1.0 & $300.01\pm0.02$ & $-8465.1\pm0.5$ & $1614.2\pm0.1$ & $807.4\pm0.1$ & $806.8\pm0.1$ & $62\pm2$ & $0.0328$ \\
2.0 & $300.01\pm0.03$ & $-8450.8\pm0.4$ & $1614.2\pm0.2$ & $809.4\pm0.1$ & $804.7\pm0.1$ & $73\pm2$ & $0.597$ \\
3.0 & $299.97\pm0.04$ & $-8425.7\pm0.5$ & $1614.0\pm0.2$ & $812.8\pm0.1$ & $801.2\pm0.2$ & $90\pm2$ & $3.44$ \\
4.0 & $299.93\pm0.04$ & $-8391.4\pm0.6$ & $1613.7\pm0.2$ & $817.1\pm0.2$ & $796.6\pm0.2$ & $119\pm2$ & $5.64$ \\

\hline
\multicolumn{8}{c}{Alternative Method} \\
1.0 & $300.00\pm0.02$ & $-8465.5\pm0.5$ & $1614.1\pm0.1$ & $807.1\pm0.1$ & $807.0\pm0.1$ & $65\pm2$ & $0.108$ \\
2.0 & $299.99\pm0.03$ & $-8450.5\pm0.5$ & $1614.1\pm0.2$ & $809.3\pm0.1$ & $804.7\pm0.1$ & $72\pm2$ & $1.31$ \\
3.0 & $299.98\pm0.04$ & $-8425.9\pm0.5$ & $1614.0\pm0.2$ & $812.5\pm0.1$ & $801.5\pm0.2$ & $88\pm2$ & $2.53$ \\
4.0 & $300.01\pm0.04$ & $-8390.4\pm0.6$ & $1614.2\pm0.2$ & $817.4\pm0.2$ & $796.8\pm0.2$ & $120\pm2$ & $7.31$ \\

\hline
\multicolumn{8}{c}{Method of Kamberaj \textit{et al}.\cite{Kamberaj2005}} \\
1.0 & $299.53\pm0.02$ & $-8470.9\pm0.5$ & $1611.6\pm0.1$ & $806.4\pm0.1$ & $805.2\pm0.1$ & $52\pm2$ & $0.023$ \\
2.0 & $298.24\pm0.03$ & $-8471.2\pm0.5$ & $1604.6\pm0.2$ & $805.4\pm0.1$ & $799.2\pm0.1$ & $47\pm2$ & $17.0$ \\
3.0 & $295.94\pm0.04$ & $-8480.9\pm0.6$ & $1592.3\pm0.2$ & $804.2\pm0.1$ & $788.1\pm0.1$ & $28\pm2$ & $53.2$ \\
4.0 & $292.72\pm0.04$ & $-8494.7\pm0.7$ & $1575.0\pm0.2$ & $802.4\pm0.1$ & $772.5\pm0.2$ & $-1\pm2$ & $60.6$ \\

\hline
\multicolumn{8}{c}{Double-Thermostat Analogue of the Proposed Method} \\
1.0 & $300.00\pm0.02$ & $-8466.0\pm0.5$ & $1614.1\pm0.1$ & $806.6\pm0.1$ & $807.5\pm0.1$ & $60\pm2$ & $6.42$\\
2.0 & $299.98\pm0.03$ & $-8452.0\pm0.5$ & $1614.0\pm0.2$ & $807.1\pm0.1$ & $806.9\pm0.1$ & $71\pm2$ & $80.4$ \\
3.0 & $300.16\pm0.04$ & $-8426.1\pm0.5$ & $1615.0\pm0.2$ & $808.4\pm0.1$ & $806.6\pm0.1$ & $90\pm2$ & $1.38\times10^4$ \\
4.0 & $300.75\pm0.04$ & $-8385.3\pm0.6$ & $1618.2\pm0.2$ & $811.6\pm0.1$ & $806.6\pm0.2$ & $125\pm2$ & $4.29\times10^4$ \\

\hline
\multicolumn{8}{c}{Double-Thermostat Analogue of the Alternative Method} \\
1.0 & $299.98\pm0.02$ & $-8465.6\pm0.5$ & $1614.0\pm0.1$ & $806.6\pm0.1$ & $807.4\pm0.1$ & $56\pm2$ & $4.87$ \\
2.0 & $299.96\pm0.03$ & $-8452.0\pm0.5$ & $1613.9\pm0.2$ & $807.1\pm0.1$ & $806.8\pm0.1$ & $71\pm2$ & $85.1$ \\
3.0 & $300.20\pm0.04$ & $-8426.0\pm0.5$ & $1615.2\pm0.2$ & $808.5\pm0.1$ & $806.7\pm0.1$ & $89\pm2$ & $1.50\times10^4$ \\
4.0 & $300.75\pm0.04$ & $-8384.7\pm0.6$ & $1618.1\pm0.2$ & $811.6\pm0.1$ & $806.6\pm0.2$ & $123\pm2$ & $4.33\times10^4$ \\

\hline
\multicolumn{8}{c}{Single-Thermostat Analogue of the Method of Kamberaj \textit{et al}.\cite{Kamberaj2005}} \\
1.0 & $299.52\pm0.02$ & $-8470.2\pm0.5$ & $1611.6\pm0.1$ & $805.9\pm0.1$ & $805.6\pm0.1$ & $58\pm2$ & $0.272$ \\
2.0 & $298.18\pm0.03$ & $-8471.8\pm0.5$ & $1604.3\pm0.2$ & $804.4\pm0.1$ & $799.9\pm0.1$ & $44\pm2$ & $0.390$ \\
3.0 & $295.97\pm0.04$ & $-8474.6\pm0.5$ & $1592.4\pm0.2$ & $801.6\pm0.1$ & $790.8\pm0.2$ & $29\pm2$ & $3.00$ \\
4.0 & $292.65\pm0.04$ & $-8478.7\pm0.6$ & $1574.6\pm0.2$ & $797.6\pm0.2$ & $777.0\pm0.2$ & $10\pm2$ & $8.12$ \\

\end{tabular}
\end{ruledtabular}
\end{table*}




In Fig.~\ref{fig:checkensemble} we show the result from the linear regression analysis for the potential energies sampled using Eq.~\eqref{eq:trotter_splitting_NHC}. In figure legends we show the averages temperatures simulated. It can be seen that the mentioned algorithm generates a distribution consistent with the canonical ensemble, and that is also true for the kinetic energies. Likewise, in Fig.~\ref{fig:mbdistribution} we observe a good correspondence between the kinetic energy distribution at $T= 304.99 K$ and the corresponding Maxwell-Boltzmann distribution We arrive at the same conclusions for the approach given by Eq.~\eqref{eq:modified_splitting}.

\begin{figure}
	\includegraphics{checkensemble}
	\caption{Linear plot of the log ratio probabilities with the true slope (solid line) and the measured slope from the linear regression analysis (symbols). Simulation of liquid water using Eq.~\eqref{eq:trotter_splitting_NHC}, $\Delta t$ = 1~fs, $\bar{T}_1$ = 296.03~K, $\bar{T}_2$ = 304.99~K.}
	\label{fig:checkensemble}
\end{figure}

\begin{figure}
	\includegraphics{maxwell-botlzamm-paper}
	\caption{Probability distribution of the kinetic energy with sampled values and those calculated with Eq.~\eqref{eq:mb}. Simulation of liquid water using Eq.~\eqref{eq:trotter_splitting_NHC}, $\Delta t$ = 1~fs, $\bar{T}$ = 304.99~K.}
\label{fig:mbdistribution}
\end{figure}

Finally, in Table~\ref{table:ensemblevalidation} we present the results of the maximum likelihood approach applied for both NVT integration schemes. It can be seen that the deviations from the true slope, $\Delta T$ = 8.970~K, are less than $1\sigma$. This means that effectively, the sampled values do not deviate from the correct distribution to a statistically noticeable level.

\begin{table}
	\begin{threeparttable}
		\caption{Ensemble Validation of NVT algorithms for liquid water using the Maximum Likelihood Approach \tnote{a} \tnote{b}}
		\label{table:ensemblevalidation}
		\begin{ruledtabular}
			\begin{tabular}{ccccc}
				& \multicolumn{4}{c}{true $\Delta T$ = 8.970 K} \\
				\cline{2-5}
				& \multicolumn{2}{c}{potential} & \multicolumn{2}{c}{kinetic}\\
				\hline
				Method  &$\Delta T/K$ & $\sigma$ dev & $\Delta T/K$ & $\sigma$ dev \\
				\hline % inserts single-line 
				Eq.~\eqref{eq:trotter_splitting_NHC} & 9.022 $\pm$ 0.117 & 0.57 & 8.928 $\pm$ 0.051 & 0.63 \\
				Eq.~\eqref{eq:modified_splitting}    & 8.965 $\pm$ 0.101 & 0.05 & 8.967 $\pm$ 0.052 & 0.14
		\end{tabular}
		\end{ruledtabular}
		\begin{tablenotes}
			\item[a] The standard deviation of the mean temperature for all cases is $\sigma$ = 0.03.
			\item[b] $\bar{T}_1$ = 296.03~K and $\bar{T}_2$ = 304.99~K.
		\end{tablenotes}
	\end{threeparttable}
\end{table}

%\begin{center}
%  \label{fig:pressure}
%  \includegraphics{pressure2}
%  \captionof{figure}{Histograms of pressure obtained in this work (filled bars) and using LAMMPS %(unfilled bars). Simulation of liquid water with NVE integration. $\Delta t : 1 fs$  }
%\end{center}


\section{Concluding Remarks}
\label{sec:conclusion}
Nevertheless, as will be shown in the forthcoming section, those numerical schemes successfully achieve the specified temperature and an acceptable long-term conservation of the extended energy $H$. 
by contrast
\begin{acknowledgments}
The authors acknowledge the financial support provided by Petrobras (project code CENPES 16113).
\end{acknowledgments}

\bibliography{rigid_bodies}
%\begin{table*}
%	\caption{Results with Exact Solution for Free Rotations}
%	\label{table:NVE}
%	\begin{ruledtabular}
%		\begin{tabular}{cccccccc}
%			$\Delta t$ (fs) & $T$ (K) & $\langle U\rangle$ (kcal/mol) & $\langle K\rangle$ (kcal/mol) & $\langle K_t\rangle$ (kcal/mol) & $\langle K_r\rangle$ (kcal/mol) & P (atm) & $R$ (kcal/mol.ns) \\
%			\hline
%			1.0 & $	305.91	\pm	0.03	$ & $	-8395.1	\pm	0.1	$ & $	1645.9	\pm	0.1	$ & $	823.2	\pm	0.1	$ & $	822.7	\pm	0.1	$ & $	146	\pm	2	$ & $0.0456$ \\
%			2.0 & $	305.21	\pm	0.02	$ & $	-8389.5	\pm	0.1	$ & $	1642.1	\pm	0.1	$ & $	823.0	\pm	0.1	$ & $	819.1	\pm	0.1	$ & $	149	\pm	2	$ & $0.656$ \\
%			3.0 & $	304.30	\pm	0.03	$ & $	-8375.4	\pm	0.9	$ & $	1637.2	\pm	0.1	$ & $	824.0	\pm	0.1	$ & $	813.3	\pm	0.1	$ & $	156	\pm	2	$ & $2.87$ \\
%			4.0 & $	304.5	\pm	0.5	$ & $	-8339	\pm	6	$ & $	1638	\pm	3	$ & $	829	\pm	1	$ & $	808.8	\pm	0.6	$ & $	180	\pm	2	$ & $10.1$ \\
%		\end{tabular}
%	\end{ruledtabular}
%\end{table*}

%\begin{table*}
%\caption{Results.....}
%\label{table:NVE}
%\begin{ruledtabular}
%\begin{tabular}{cccccccc}
%$\Delta t$ (fs) & $T$ (K) & $\langle U\rangle$ (kcal/mol) & $\langle K\rangle$ (kcal/mol) & $\langle K_t\rangle$ (kcal/mol) & $\langle K_r\rangle$ (kcal/mol) & P (atm) & $R$ (kcal/mol.ns) \\
%\hline
%1.0 & $305.95\pm0.03$ & $-8395.4\pm0.1$ & $1646.1\pm0.1$ & $823.3\pm0.1$ & $822.9\pm0.1$ & $151\pm2$ & $0.0471$ \\
%2.0 & $305.17\pm0.03$ & $-8389.0\pm0.1$ & $1641.9\pm0.1$ & $823.2\pm0.1$ & $818.8\pm0.1$ & $148\pm2$ & $0.853$ \\
%3.0 & $304.34\pm0.03$ & $-8374\pm2$ & $1637.5\pm0.2$ & $824.0\pm0.1$ & $813.4\pm0.1$ & $152\pm2$ & $3.96$ \\
%4.0 & $303.6\pm0.4$ & $-8349\pm5$ & $1633\pm2$ & $826.8\pm0.2$ & $806.5\pm0.4$ & $168\pm2$ & $8.25$ \\
%\end{tabular}
%\end{ruledtabular}
%\end{table*}
%\begin{table}
%	\begin{threeparttable}
%		\caption{Average deviation of the conserved quantity ($DE$) and average temperatures from liquid water simulation with NVT integration \tnote{a}\tnote{b}}
%		\label{table:denvt}
%		\begin{ruledtabular}
%			\begin{tabular}{ccccc}
%				& \multicolumn{2}{c}{Eq.~\ref{eq:trotter_splitting_NHC}} & \multicolumn{2}{c}{Eq.~%\ref{eq:modified_splitting}} \\
%				\cline{2-5}
%				$\Delta t$/fs & $D\!E$ & $T$/K & $D\!E$ & $T$/K \\
%				\hline
%				1.0 & 0.000043 & 298.16  & 0.000052  & 298.15 \\
%				2.0 & 0.00013  & 298.23  & 0.00017   & 298.06 \\
%				3.0 & 0.00089  & 298.17  & 0.0013    & 298.12
%			\end{tabular}
%		\end{ruledtabular}
%		\begin{tablenotes}
%			\item[a] The standard deviation of the mean temperature for all cases is $\sigma = 0.037$.
%			\item[b] The smaller time step in the NHC and NHC* operators is 0.25 fs.
%		\end{tablenotes}
%	\end{threeparttable}
%\end{table}

%The Nos\'e-Hoover chain propagator is further split by making $i\!L_\text{NHC} = \sum_{j=0}^M i\!L_%\text{NHC}^{(j)}$ and, afterwards,
%\[
%e^{\frac{\Delta t}{2} i\!L_\text{NHC}} = \Pi_{j=1}^n \Bigg[ \prod_{j=M}^1 e^{\frac{\Delta t}{4} i\!L_\text{NHC}^{(j)} } \Bigg] e^{\frac{\Delta t}{2} i\!L_\text{NHC}^{(0)} } \Bigg[ \prod_{j=1}^M e^{\frac{\Delta t}{4} i\!L_\text{NHC}^{(j)} } \Bigg].
%\]

%In order to satisfy the criterion in Eq.~\eqref{eq:split_kappa_w} for all factors, given the function $w$ expressed in Eq.~\eqref{eq:nhc_measure}, we further split the NHC contribution as $i\!L_\text{NHC} = \sum_{j=0}^M i\!L_\text{NHC}^{(j)}$, where

%\begin{subequations}
%\label{eq:solution_momenta}
%\begin{align}
%&{\vt p}_i(t) = {\vt p}_i^0 + \left({\vt F}_i - \alpha_1 {\vt p}_i^0 \right) \phi\left(\alpha_1 t \right) %t \quad \text{and} \label{eq:solution_p} \\
%&{\vt \pi}_i(t) = {\vt \pi}_i^0 + \left(2 {\mt C}_i{\vt \tau}_i - \alpha_1 {\vt \pi}_i^0 \right) \phi%\left(\alpha_1 t \right) t. \label{eq:solution_pi}
%\end{align}
%\end{subequations}

%By evaluating these functions simultaneously for all $i$, we accomplish the effect of a propagator $e^{t i%\!L_b^\ast}$, where $i\!L_b^\ast$ is the Liouville operator given by
%\[
%i\!L_b^\ast = \sum_{i=1}^N \left[ \left( \tr{\vt F}_i - \alpha_1 \tr{\vt p}_i \right) \diff{}{\vt p_i} + \left(2 \tr{\vt \tau}_i \tr{\mt C}_i - \alpha_1 \tr{\vt \pi}_i \right) \diff{}{\vt \pi_i} \right].
%\]
%We also define a modified Nos\'e-Hoover chain propagator $e^{i\!L^\ast_\text{NHC}}$, as follows:
%\begin{equation}
%\begin{split}
%e^{\frac{\Delta t}{2} i\!L_\text{NHC}^\ast } = &\prod_{j=M}^1 \exp\left[\frac{\Delta t}{4} \left( G_j - \alpha_{j+1} p_{\eta_j} \right) \diff{}{p_{\eta_j}}\right] \\
%\times &\exp\left(-\frac{\Delta t}{2} \sum_{j=1}^M \alpha_j \diff{}{\eta_j}\right) \\
%\times &\prod_{j=1}^M \exp\left[\frac{\Delta t}{4} \left( G_j - \alpha_{j+1} p_{\eta_j} \right) \diff{}{p_{\eta_j}}\right]
%\end{split}
%\end{equation}

%In order to evaluate the stability of the numerical integrators, we employ a measure given by%\cite{Tuckerman2010}
%\begin{equation}
%\label{eq:performance}
%D\!E =  \frac{1}{\mathcal N} \sum_{t=1}^{\mathcal N} \left| \frac{E_t - E_0}{E_0} \right|,
%\end{equation}
%from which it is possible to quantify the average deviation of the supposedly conserved quantity $E$ %(either $\mathcal{H}$ of Eq.~\eqref{eq:H_NVE} or $H$ of Eq.~\eqref{eq:H_nvt}) from its initial value $E_0$ %after a certain number of time steps $\mathcal N$.
% \textbf{In order to use a constant value for the smaller time step in the NHC and NHC* operators $(\Delta t/4)$, we factorized those operators analogously to the scheme given by Eq.~\eqref{eq:splitting_rot}}. 


%\begin{equation}
%\label{eq:probability_ratio_pot}
%\frac{P_{pot}(E/\beta_2)}{P_{pot}(E/\beta_1)} = \frac{Q_{pot}(\beta_1)}{Q_{pot}(\beta_2)}  \exp\left[(-\beta_2 - \beta_1)E_{pot}\right]
%\end{equation}

%\begin{equation}
%\label{eq:probability_ratio_kin}
%\frac{P_{kin}(E/\beta_2)}{P_{kin}(E/\beta_1)} = \frac{Q_{kin}(\beta_1)}{Q_{kin}(\beta_2)}  \exp\left[(-%\beta_2 - \beta_1)E_{kin}\right]
%\end{equation}
\end{document}
